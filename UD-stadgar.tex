\documentclass[a4paper]{article}
\usepackage[top=1.5in,bottom=1.5in,left=1.7in,right=1.7in,headsep=0.15in,a4paper]{geometry}
\usepackage[T1]{fontenc}
\usepackage[swedish]{babel}
\usepackage[utf8]{inputenc}
\usepackage{graphicx}
\usepackage{amsmath}
\usepackage{titlesec}
\usepackage{fancyhdr}
\titlelabel{\llap{\S\thetitle\quad}}

\titleformat{\section}[block]
{\Large\bfseries}%
{\llap{\S\thesection\quad}}
{0mm}
{}

\titleformat{\subsection}[block]
{\large\bfseries}%
{\llap{\S\thesubsection\quad}}
{0mm}
{}

\lhead{Uppsala Datavetare}
\chead{}
\rhead{2018-11-28}
\lfoot{}
\cfoot{}
\rfoot{}
\pagestyle{fancy}
\begin{document}
\begin{titlepage}
  \vspace*{\fill}
  \centering
  \vfill
  \vfill
  \includegraphics[width=\textwidth]{UD_center.png}
  {\huge\textbf{Stadgar}\\
    \vspace{0.3em}
    \large{Fastställda vid stormöte 2014-03-26\\
      Reviderade vid stormöte 2014-11-12, 2015-10-28, 2017-11-20, 2018-11-28, 2021-11-25}}
  \vfill
  \vfill
  \vspace*{\fill}
\end{titlepage}
\renewcommand{\contentsname}{Innehåll\hfill\small Sida}
\tableofcontents
\cleardoublepage 
\setcounter{page}{1}
\cfoot{\thepage}
\section{ALLMÄNNA BESTÄMMELSER}
{\subsection{Namn och föreningsform}
  Föreningens namn är Uppsala Datavetare, förkortas till UD. Uppsala Datavetare är en politiskt och religiöst obunden ideell förening.
  \subsection{Föreningens säte}
  Föreningens styrelse har sitt säte i Uppsala kommun.
  \subsection{Sammansättning}
  Föreningen representerar och består av studenter vid kandidatprogrammet i datavetenskap, masterprogrammet i datavetenskap och masterprogrammet i människa-datorinteraktion vid Uppsala universitet, hädanefter kallade datavetare.
  \subsection{Verksamhet och målsättning}
  Föreningen företräder datavetare vid Uppsala universitet. Föreningen ska tillvarata, främja och bevaka dessa studenters intressen, främst vad avser utbildningsfrågor och studiesociala frågor. Därutöver har föreningen i uppgift att tillhandahålla informativ och social service för dessa studenter samt att verka för ett gott kamratskap dem emellan.
  \subsection{Tillhörighet}
  Föreningen representerar DV-sektionen i Uppsala teknolog- och naturvetarkår som dess sektionsförening.
  \subsection{Firmateckning}
  Föreningen firmatecknas var för sig av ordförande samt vice ordförande.
  \subsection{Överklagande}
  Överklagan skall inkomma till styrelsen för att bedömas om den är av en sådan karaktär, att ett stormöte måste kallas för att behandla den sagda överklagan. Om detta inte finns skall styrelsen fatta beslutet om överklagan vinner kraft, om styrelsen kan anses jävig skall en icke jävig arbetsgrupp tillsättas för att utreda sagda överklagan sedanmera presentera sitt utfall till styrelsen.
  \subsection{Undanröjande}
  Beslut taget av någon av föreningen eller av föreningen utsedd kan av stormötet undanröjas  endast om det uppenbart strider mot §1.4.
  \section{FÖRENINGENS MEDLEMMAR}
  {\subsection{Definition}
    Föreningens styrelse skall anses som föreningens officiella medlemmar.
    \subsection{Skyldigheter}
    Medlem och datavetare är skyldig:
    \begin{itemize}
    \item att följa föreningens stadgar och reglemente
    \end{itemize}
    \subsection{Rättigheter}
    Medlem och datavetare som fullgjort sina skyldigheter enligt \S2.2 äger rätt:
    \begin{itemize}
    \item att deltaga i sammankomster som anordnas för datavetare
    \item att tillhandages information om föreningens angelägenheter
    \item att äga närvarorätt vid styrelsemöten
    \item att med en röst deltaga i val av poster i föreningen samt val av representanter som företräder föreningen
    \item att närvara med yttranderätt, yrkanderätt och rösträtt vid föreningens stormöten
    \item att taga del av protokoll och övriga handlingar som berör föreningen
    \item tt kandidera vid allmänna val inom föreningen
    \end{itemize}
    \subsection{Uteslutning}
    Datavetare får inte uteslutas ur föreningens verksamhet av annan anledning än att denne har försummat att betala av föreningen beslutade avgifter, motarbetat föreningens verksamhet eller ändamål, eller skadat föreningens intressen.\\
    Beslut om uteslutning får begränsas till att omfatta viss tid. Sådan tidsbegränsad uteslutning får som mest omfatta sex månader från beslutsdagen.\\
    Om tillräckliga skäl för uteslutning inte föreligger får föreningen istället meddela datavetaren en varning.\\
    Beslut om uteslutning eller varning får inte fattas utan att datavetaren inom 14 dagar fått tillfälle att yttra sig över de omständigheter som föranlett att frågan om uteslutning tagits upp. I beslutet skall skälen härför redovisas samt anges vad datavetaren skall iaktta för överklagande. Beslutet skall inom tre dagar från dagen för beslutet skriftligen tillställas den berörde.\\
    Beslut om uteslutning eller varning skall enhälligt fattas av föreningens styrelse.}
  \section{FÖRENINGENS HEDERSMEDLEMMAR}
  {\subsection{Definition}
    Personer som valts till hedersmedlem av föreningen.
    \subsection{Val}
    Hedersmedlem kan utses av föreningens stormöte med enhällig majoritet av samtliga avgivna röster. Person som har gjort betydande insatser för föreningens verksamhet, existens eller fortlevande kan nomineras till hedersmedlem.
    \subsection{Skyldigheter}
    Hedersmedlem är skyldig:
    \begin{itemize}
    \item att följa föreningens stadgar och reglemente
    \end{itemize}
    \subsection{Rättigheter}
    Hedersmedlem äger rätt:
    \begin{itemize}
    \item att närvara med yttrande- och yrkanderätt vid stormöte
    \item att ta del av protokoll och övriga handlingar som berör föreningen
    \item att deltaga i sammankomster som anordnas för medlemmarna
    \item att ha tillgång till information om föreningens angelägenheter
    \end{itemize}
    \subsection{Uteslutning}
    Uteslutning av hedersmedlem kan fattas av styrelsen om denne motverkar föreningens syfte eller anda.}
  \section{VERKSAMHETSÅR}
  {\subsection{Verksamhets- och räkenskapsår}
    Föreningens verksamhetsår och räkenskapsår omfattar tiden fr.o.m den 1 januari t.o.m den 31 december.}
  \section{STORMÖTE}
  {\subsection{Befogenhet}
  Stormötet är föreningens högst beslutande organ.
  \subsection{Rösträtt}
  Icke uteslutna datavetare har rösträtt på stormötet. Rösträtten är personlig och får inte utövas genom ombud. Rösträtt i val av ledamöter och suppleanter i Uppsala teknolog- och naturvetarkårs fullmäktige har icke utesluten datavetare som är
  medlem i Uppsala teknolog- och naturvetarkår.
  \subsection{Beslutförhet}
  Stormötet är beslutsmässigt då minst elva röstberättiga medlemmar enligt §5.2 är närvarande på mötet.  
  \subsection{Adjungeringar}
  Ständigt adjungerade med yttrande- och yrkanderätt vid stormötena tillfaller: 
  \begin{itemize}
  \item Programansvarig 
  \item Studievägledare
  \item Revisor
  \end{itemize}
  \subsection{Mötesordförande}
  Stormötets förhandlingar leds av mötet tillförordnad mötesordförande.
  \subsection{Kallelse}
  Styrelsen kallar till stormöte.\\
  Kallelse till stormöte skall senast 20 dagar före mötet tillställas medlemmarna.\\
  Kallelse till extra stormöte skall senast 15 dagar före mötet tillställas medlemmarna.\\
  Kallelse skall anslås på hemsidan, anslagstavlan och på annan lämplig plats.\\
  Höstmötet skall hållas senast den 30 november.\\
  Vårmötet skall hållas senast den 20 maj.
  \subsection{Handlingar}
  Föredragslistan samt handlingar till mötet skall anslås senast sju dagar före mötet.
  \subsection{Förslag till ärenden att behandlas av stormötet}
  Såväl medlem som styrelsen får avge förslag att behandlas av stormötet.\\
  \\
Förslag från medlem skall vara styrelsen tillhanda senast tio dagar före stormötet. Styrelsen skall till stormötet avge ett skriftligt yttrande över förslaget.
  \subsection{Ordinarie ärenden vid höstmöte}
  Höstmötet skall:
  \begin{enumerate}
  \item Fastställa röstlängd för mötet
  \item Val av ordförande och sekreterare för mötet utföras
  \item Val av protokolljusterare och två rösträknare utföras
  \item Fastställa om mötet har utlysts på rätt sätt
  \item Fastställa föredragningslista
  \item Motioner och propositioner
  \item Fastställande av verksamhetsplan samt behandling av budget för det kommande verksamhets-/räkenskapsåret
  \item Val enligt reglemente(Bilaga)
  \item Övriga frågor
  \end{enumerate}
  \subsection{Ordinarie ärenden vid vårmöte}
  Vårmötet skall:
  \begin{enumerate}
  \item Fastställa röstlängd för mötet
  \item Val av ordförande och sekreterare för mötet utföras
  \item Val av protokolljusterare och två rösträknare utföras
  \item Fastställa om mötet har utlysts på rätt sätt
  \item Fastställa föredragningslista
  \item Verksamhetsberättelse och bokslut för det senaste verksamhetsåret
  \item Revisionsberättelse för det senaste verksamhetsåret
  \item Motioner och propositioner
  \item Fråga om ansvarsfrihet för styrelsen för den tid revisionen avser
  \item Val enligt reglemente(Bilaga)
  \item Övriga frågor
  \end{enumerate}
  \subsection{Extra stormöte}
  Styrelsen kan kalla föreningen till extra stormöte.\\
  Styrelsen är skyldig att kalla till extra stormöte när en revisor eller minst femton röstberättigade datavetare begär det. Sådan framställning skall avfattas skriftligen och innehålla anledning för extra möte.\\
  När styrelsen mottagit en begäran om extra stormöte skall den inom 14 dagar utlysa sådant möte att hållas inom två månader från erhållen begäran.\\
  Vid extra stormöte får endast det som föranlett mötet upptas till behandling.\\
  Om rösträtt på extra stormöte och om beslutsmässighet vid sådant möte gäller §5.2 och §5.3.\\
  \\
  Underlåter styrelsen att utlysa eller kalla till extra stormöte får de som gjort framställningen vidta åtgärder och kalla till extra stormöte enligt föregående stycke.}
\section{VAL}
{\subsection{Valbarhet}
  Valbar till styrelsen och valberedningen är röstberättigad datavetare. Arbetstagare inom föreningen får dock inte väljas till ledamot av styrelsen, valberedningen eller till revisor i föreningen. Valbar till ledamot eller suppleant i Uppsala
  teknolog- och naturvetarkårs fullmäktige är röstberättigad datavetare som är medlem i Uppsala teknolog- och naturvetarkår.
  \subsection{Nominering}
  Stormötets val förbereds av valberedningen. Utöver valberedningens förslag må fri kandidatnominering ske tills tiden för frågans avgörande.
  \subsection{Fyllnadsval}
  Styrelsemedlemmar, revisorn och valberedningen kan endast väljas av stormötet. Representanter mot universitet och mot Uppsala teknolog- och naturvetarkår kan endast nomineras av stormötet. Fyllnadsval av övriga poster kan förrättas av styrelsen.
  \section{PROCEDURREGLER}
  \subsection{Röstning}
  Röstning via fullmakt är ej tillåten.
  \subsection{Jävighet}
  Ingen må deltaga i beslut om ansvarsfrihet för åtgärd för vilken hen är ansvarig, eller i beslut av vars utgång hen kan äga ekonomiskt intresse.
  \subsection{Acklamation och votering}
  Beslut fattas med bifallsrop (acklamation) eller om så begärs efter omröstning (votering). Val med två eller fler kandidater avgörs med sluten omrösting.\\
  Om röstberättigad medlem begär det skall omröstning ske slutet.\\
  \\
  Val med undantag för val i §3 avgörs genom relativ majoritet.\\
  För beslut i andra frågor än val krävs kvalificerad majoritet av antalet avgivna röster om inte annat anges i stadgarna.\\
  \\
  Vid omröstning med tre eller fler alternativ då lika röstetal uppstår mellan de med flest röster skall omval göras mellan enbart de alternativen.\\
  Vid omröstning med två alternativ som inte avser val gäller vid lika röstetal det förslag som biträds av ordföranden vid mötet, om hen är röstberättigad. Är hen inte röstberättigad avgör lotten. Vid val med enbart två alternativ skall i händelse av lika röstetal lotten avgöra. 
  \subsection{Adjungerade}
  Adjungerade må deltaga i sammanträden med yttrande- och yrkanderätt.}
\section{PROTOKOLL}
{\subsection{Stormöte}
  Vid stormöten ska protokoll föras, som upptar anteckningar om ärendets art, samtliga ställda och ej återtagna yrkanden, beslut, särskilda yttranden och reservationer samt förteckning över de närvarande.
  \subsection{Övriga sammanträden}
  Vid föreningens övriga sammanträden skall föras protokoll som tar upp tagna beslut och förteckning över de närvarande.
  \subsection{Justering}
  Stormötesprotokoll skall inom 7 dagar vara justerade och påskrivna av mötesordförande, mötessekreterare samt protokolljusterare.\\
  Styrelsemötesprotokoll skall innan nästkommande styrelsemöte vara justerade och påskrivna av styrelseordförande, mötessekreterare samt protokolljusterare.
  \subsection{Offentliggörande}
  Justerade protokoll skall tillgängliggöras för datavetare via hemsidan och föreningens anslagstavla.
  \subsection{Arkivering}
  Samtliga protokoll skall arkiveras.}
\section{EKONOMI}
{\subsection{Skötsel}
  Föreningens ekonomi sköts av vice ordförande tillika ekonomiskt ansvarige.
  \subsection{Redovisning}
  Styrelsen ansvarar för att föreningens redovisning sköts i enlighet med myndigheters krav och god redovisningssed.}
\section{STADGARNA}
{\subsection{Tolkning}
  Vid tolkning av stadgarna gäller styrelseordförandes tolkning in tills dess att stormötet beslutat i sak. Tolkning skall protokollföras och redovisas på nästkommande stormöte.
  \subsection{Stadgeändring}
  För ändring av stadgarna krävs beslut av stormöte med 2/3 kvalificerad majoritet av antalet avgivna röster. Förslag till ändring av stadgarna får skriftligen avges av såväl datavetare som styrelsen.}
\section{REGLEMENTE}
{\subsection{Definition}
  Reglementet är ett tillägg till stadgarna, i vilket tillämpningsföreskrifter och övriga föreskrifter återfinns. Reglementet måste vara helt i enighet med stadgarna.
  \subsection{Tolkning}
  Vid tolkning av reglementet gäller styrelseordförandes tolkning in tills dess att stormötet beslutat i sak. Tolkning skall protokollföras och redovisas på nästkommande stormöte.
  \subsection{Ändring}
  För ändring av detta reglemente krävs beslut av stormöte med 2/3 kvalificerad majoritet av antalet avgivna röster. Förslag till ändring av reglemente får skriftligen avges av såväl datavetare som styrelsen.}
\section{VALBEREDNINGEN}
{\subsection{Uppgift}
  Valberedningens uppgift är att förbereda val som skall förrättas av stormötet. Valberedningen skall också på uppdrag av styrelsen förbereda val till fyllnadsval §6.3.
  \subsection{Sammansättning}
  Valberedningen består av sammankallande samt en valberedning utsedd av den sammankallande.
  \subsection{Åligganden}
  Senast tio dagar före val skall valberedningen meddela styrelsen sina förslag.
  \subsection{Sammanträdande}
  Valberedningen sammanträder när sammankallande bestämmer.}
\section{REVISOR}
{\subsection{Uppgift}
  Revisorn skall granska styrelsens förvaltning och räkenskaper för det senaste verksamhets- och räkenskapsåret samt till styrelsen överlämna revisionsberättelse senast 14 dagar före stormötet.
  \subsection{Val}
  Datavetare eller icke-datavetare, med undantag för utesluten datavetare, kan ställa upp eller bli nominerad till val av revisor.
  \subsection{Rättigheter}
  Revisorn har rätt att fortlöpande ta del av föreningens räkenskaper, stormötes- och styrelseprotokoll, och övriga handlingar. Revisorn har närvaro- och yttranderätt vid styrelsemöten. Föreningens räkenskaper skall vara revisorn tillhanda senast 20 dagar före stormötet.}
\section{STYRELSEN}
{\subsection{Befogenheter}
  Styrelsen är föreningens högst verkställande organ.
  \subsection{Sammansättning}
  Styrelsen består av ordförande, vice ordförande, studierådsordförande samt minst fyra ledamöter vars arbetsuppgift anges i reglementet.\\
  Styrelsen får utse person till ständigt adjungerad ledamot. Sådan ledamot har inte rösträtt men kan efter beslut av styrelsen ges förslagsrätt. Hen får utses till befattning inom styrelsen. Klubbmästaren som röstades in för två stormöten sedan har vägande röst vad gäller klubbmästarens röst i styrelsen.
  \subsection{Beslutsmässighet}
  Styrelsen är beslutsmässig när samtliga styrelsemedlemmar kallats och då minst halva antalet styrelsemedlemmar är närvarande.\\
  I brådskande fall får ärenden beslutas per capsulam och skall avgöras genom skriftlig omröstning, vid telefonsammanträde eller över IRC. Om särskilt protokoll inte upprättas skall sådant beslut anmälas vid det närmast därefter följande sammanträde.
  \subsection{Styrelsens åligganden}
  När stormöte inte är samlat är styrelsen föreningens beslutande organ och ansvarar för föreningens angelägenheter. Styrelsen skall svara för föreningens verksamhet enligt fastställda planer samt tillvarata datavetarnas intressen.\\
  Det åligger styrelsen särskilt:
  \begin{itemize}
  \item att tillse att för föreningen gällande lagar och bindande regler iakttas
  \item att verkställa av stormötet fattade beslut
  \item att planera, leda och fördela arbetet inom föreningen
  \item att ansvara för och förvalta föreningens medel
  \item att tillställa revisorn räkenskaper och övriga handlingar enligt §13.3
  \item att förbereda stormöte
  \end{itemize}
  Ordförande är föreningens officiella representant och bör närvara på relevanta möten. Ordförande skall leda styrelsens förhandlingar och arbete samt övervaka att föreningens stadgar och att övriga för föreningen bindande regler och beslut efterlevs.\\
  \\
  Styrelsen skall arbeta efter arbetsbeskrivningar i reglementet.
  \subsection{Kallelse}
  Styrelsen sammanträder på kallelse av ordföranden, eller då minst halva antalet styrelsemedlemmar har begärt det.
  \subsection{Solidaritet}
  Styrelseledamot som utan reservation deltagit i beslut som fattats av styrelsen är solidariskt ansvariga för detta. Styrelseledamot som ej närvarat vid beslut är solidariskt ansvarig om hen inte reserverat sig i protokoll senast vid nästa sammanträde då hen varit närvarande.}
\section{UTSKOTT}
{\subsection{Definition}
  Ett utskott inom föreningen är en av styrelsen erkänd separat grupp som har kontinuerliga återkopplingar med föreningens styrelse.
  \subsection{Åligganden}
  Det åligger föreningens utskott att följa föreningens stadgar och reglemente.}
\section{UPPLÖSNING}
{\subsection{Upplösning}
  För upplösning av föreningen krävs beslut av två på varandra följande stormöten med 2/3 kvalificerad majoritet av antalet avgivna röster. I beslut om upplösning av föreningen skall anges var den upplösta föreningens handlingar m.m. skall arkiveras, till exempel i folkrörelsearkiv eller motsvarande.
  \subsection{Tillgångar}
  Vid upplösning skall föreningens tillgångar sättas under förvaltning av den studentkår som omfattar datavetarna vid Uppsala universitet och tillfalla en eventuell ny organisation med samma eller liknande mål som föreningen. Om ingen sådan organisation uppstått efter fem års förvaltning tillfaller pengarna ovan nämnda studentkår, att användas för utbildningsbevakning.}
\end{document}
