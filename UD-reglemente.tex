\documentclass[a4paper]{article}
\usepackage[top=1.5in,bottom=1.5in,left=1.7in,right=1.7in,headsep=0.15in,a4paper]{geometry}
\usepackage[T1]{fontenc}
% \usepackage[swedish]{babel}
\usepackage[utf8]{inputenc}
\usepackage{graphicx}
\usepackage{amsmath}
\usepackage{titlesec}
\usepackage{fancyhdr}
\titlelabel{\llap{\S\thetitle\quad}}

\titleformat{\section}[block]
{\Large\bfseries}%
{\llap{\S\thesection\quad}}
{0mm}
{}

\titleformat{\subsection}[block]
{\large\bfseries}%
{\llap{\S\thesubsection\quad}}
{0mm}
{}

\lhead{Uppsala Datavetare}
\chead{}
% TODO: Uppdatera raden under varje gång dokumentet uppdateras
\rhead{2020-04-01}
\lfoot{}
\cfoot{}
\rfoot{}
\pagestyle{fancy}
\begin{document}
\begin{titlepage}
  \vspace*{\fill}
  \centering
  \vfill
  \vfill
  \includegraphics[width=\textwidth]{UD_center.png}
  {\huge\textbf{Reglemente}\\
    \vspace{0.3em}
% TODO: Uppdatera raden under varje gång dokumentet uppdateras
    \large{Fastställda vid stormöte 2014-03-26\\
      Reviderade vid stormöte 2014-11-12, 2015-10-28, 2016-04-20, 2017-11-20,
      2018-05-15, 2018-11-28, 2019-05-23, 2019-11-25, 2020-05-13}}
  \vfill
  \vfill
  \vspace*{\fill}
\end{titlepage}
\renewcommand{\contentsname}{Innehåll\hfill\small Sida}
\tableofcontents
\cleardoublepage 
\setcounter{page}{1}
\cfoot{\thepage}
\section{Föreningsaktiva}
{Det åligger samtliga föreningsaktiva:
  \begin{itemize}
    \item att utföra sina uppgifter efter bästa förmåga
    \item att representera föreningen i en positiv anda utåt
    \item att göra ett ordentligt överlämnande till sina efterträdare
    \item att skriftligen lämna rekommendationer och erfarenheter till sina efterträdare
  \end{itemize}}

\section{Föreningsstyrelsen}
{\subsection{Ordförande}
  Ordföranden är föreningens officiella representant och skall närvara på relevanta möten. Ordföranden skall leda styrelsens förhandlingar och arbete samt övervaka att föreningens stadgar och övriga för föreningen bindande regler och beslut efterlevs.\\
  Ordförande är även ansvarig för:
  \begin{itemize}
    \item att förbereda styrelsens sammanträden och föreningens möten tillsammans med sekreteraren
    \item att närvara på UTN:s ordföranderåd
    \item att uppehålla kontakten med relevanta parter
    \item att representera sektionen i UTNs fullmäktige i fallet att ingen annan representant har valts
  \end{itemize}
  
  \subsection{Vice ordförande tillika ekonomiskt ansvarig}
  Vice ordförande är ansvarig för:
  \begin{itemize}
    \item att svara för föreningens bokföring
    \item att årligen upprätta balans- samt resultaträkningar
    \item att utarbeta underlag för budget och budgetuppföljning
    \item att se till att föreningens skatter, avgifter och skulder betalas i rätt tid
    \item att i förekommande fall upprätta och avge allmän självdeklaration, särskild uppgift, kontrolluppgifter, skattedeklaration och övriga föreskrivna uppgifter inom skatte- och avgiftsområdet
    \item att föra inventarieförteckning då detta är applicerbart
    \item att närvara på UTN:s ekonomiråd
    \item att ansvara för inköp till föreningen
    \item att vara suppleant för att representera sektionen i UTNs fullmäktige i fallet att ingen annan representant har valts
  \end{itemize}
  
  \subsection{Sekreterare}
  Sekreterare är ansvarig för:
  \begin{itemize}
    \item att förbereda styrelsens sammanträden och föreningens möten tillsammans med ordföranden
    \item att föra protokoll över styrelsens sammanträden
    \item att se till att föreningens handlingar hålls ordnade och förvaras på ett betryggande sätt samt ansvara för att föreningens historia dokumenteras
    \item att se till att fattade beslut har verkställts
    \item att offentliggöra protokoll för datavetare
    \item att om ordföranden inte bestämmer annat, underteckna utgående handlingar
    \item att årligen ansvara för att förslag till verksamhetsberättelse för föreningen upprättas
    \item att vara suppleant för att representera sektionen i UTNs fullmäktige i fallet att ingen annan representant har valts
  \end{itemize}
  
  \subsection{Studierådsordförande}
  Studierådsorförande är ansvarig för:
  \begin{itemize}
    \item att sammankalla till regelbundna studierådsmöten
    \item att närvara på programrådet, det vill säga vara en av studentrepresentanterna i programrådet
    \item att vara ombud till UTN:s utbildningsutskott
    \item att representera sektionen i UTNs fullmäktige i fallet att ingen annan representant har valts
    \item att tillhöra och leda studirådet
  \end{itemize}
  
  \subsection{Klubbmästare}
  Klubbmästare är ansvarig för:
  \begin{itemize}
    \item att sammankalla till regelbundna klubbverkssammanträden
    \item att vara ansvarig för att anordna traditionsenliga fester och andra eventuella evenemang
    \item att närvara på UTN:s klubbmästarråd
    \item att tillhöra och leda klubbverket
  \end{itemize}
  
  \subsection{Tillgångsansvarig}
  Tillgångsansvarig är ansvarig för:
  \begin{itemize}
    \item att förvalta föreningens fysiska tillgångar
    \item att hålla förrådet välorganiserat
    \item att administrera föreningens nycklar
    \item att sköta kioskverksamheten i FooBar
    \item att tillhöra och leda tillgångsutskottet
  \end{itemize}
  
  \subsection{Informationsansvarig}
  Informationsasvarig är ansvarig för:
  \begin{itemize}
    \item att administrera relevanta informationsresurser
    \item att ansvara för att arrangemang blir marknadsförda genom relevanta informationkanaler
    \item att agera företagskontakt för föreningen
    \item att hålla kontakt med PR-amanuensen
    \item att tillhöra och leda informationsutskottet
  \end{itemize}}

\subsection{Studiesocialt Ansvarig}
Studiesocialt ansvarig är ansvarig för:
\begin{itemize}
  \item att föra sektionsmedlemarnas studiesociala åsikter vidare
  \item att värna om arbetsmiljön i sektionsrummet
  \item att vara sektionens huvudskyddsombud
  \item att tillsätta skyddsombud fördelade över sektionen
  \item att närvara på UTNs studiesociala utskott
  \item att anordna studiesociala aktiviteter för föreningen
  \item att vara sektionens representant till institutionens arbetsmiljögrupp
\end{itemize}

\section{Stormöte och informationsmöte}
\subsection{Informationsmöte}
  Informationsmötet skall:
  \begin{itemize}
    \item ligga så tidigt på hösten som möjligt
    \item informera om vilka UD, studierådet och klubbverket är och vad de gör
    \item försöka rekrytera någon ny student till studierådet
  \end{itemize}
  \subsection{Höstmöte}
  Höstmötet skall genomföra val av:
  \begin{itemize}
    \item styrelse inför kommande verksamhetsår med undantag för en av två klubbmästare om sådan blev vald under det senaste vårmötet och därmed ingår i styrelsen under kommande vårtermin
    \item en revisor för en tid av 1 år
    \item ombud till programrådet för kandidatprogrammet (2 ordinarie + 1 suppleant)
    \item ombud till programrådet för masterprogrammet i datavetenskap
    \item representant till institutionen för informationsteknologi för en tid av 1 år (1)
    \item ombud och suppleanter till UTN:s fullmäktige
    \item ombud till UTN:s valberedning (1)
    \item ombud till UTN:s internationella utskott (1)
    \item ombud till UTN:s arbetsmarknadsutskott (1)
    \item kandidater till övriga utskott där datavetare skall representeras
    \item sammankallande till valberedningen
  \end{itemize}
  \subsection{Vårmöte}
  Vårmötet skall genomföra val av:
  \begin{itemize}
    \item sammankallande till valberedningen
    \item representant till institutionens lika villkorsgrupp
    \item ytterligare klubbmästare inför kommande två terminer
    \item sportansvarig
    \item ombud till programrådet för masterprogrammet i dataanalys (1 ordinarie)
    \item ombud till programrådet för masterprogrammet i bildanalys och maskininlärning (1 ordinarie)
  \end{itemize}
  samt:
  \begin{itemize}
    \item genomföra eventuella fyllnadsval
    \item framföra årsredovisning av föregående verksamhetsår
    \item behandla beslut om ansvarsfrihet för föregående styrelse samt avgående
      klubbmästarens arbete för föregående termin
    \item behandla beslut om att överlämna den avgående klubbmästarens ansvar
      över dennes arbete sedan räkenskapsårets start till kvarsittande styrelse
  \end{itemize}

\section{Utskott}
{\subsection{Studieråd}
  Studierådet består av studierådsordförande samt ledamöter. Studierådet skall eftersträva en jämn könsfördelning och då detta är applicerbart bestå av minst en representant från varje årskurs.\\
  Studierådet ansvarar för utbildningsbevakning och studiesociala frågor för studenterna och verkar därmed för granskning av innehåll i utbildningen. Studierådet ska även verka för att bevaka studenternas intressen med avseende på studiemiljö.\\
  Studierådet skall också ansvara för att i samråd med föreningens styrelse se till att det datavetenskapliga programmens studenter finns representerade i alla relevanta utbildningsbevakande instanser, bland annat UTN:s utbildningsutskott.\\
  Studierådet utser inom sig de övriga befattningshavare som behövs.
  
  \subsection{Informationsutskottet}
  Informationsutskottet är ansvarig för:
  \begin{itemize}
    \item att hemsidan hålls uppdaterad med relevant information
    \item att administrera Facebook-gruppen “Sektionen Uppsala Datavetare”
    \item att administrera mailinglistan “Uppsala Datavetare” via Mailchimp
    \item att information om alla evenemang når ut till studenterna
    \item att hemsidan finns tillgänglig på engelska
    \item att informera studenterna om alla relevanta föreningar och sammanslutningar
    \item att postbeskrivningar för samtliga poster finns på hemsidan
    \item att anordna foto på hela föreningsstyrelsen
    \item att ansvara för att märken trycks (Capseisa, Styrelsen, Sjöslaget osv)
    \item att anordna allt kring sektionsflotten till forsränningen
    \item att ansvara för marknadsföringsmaterial (Tröjor, väskor, muggar, osv)
  \end{itemize}
  
  \subsection{Datavetenskapliga Klubbverket}
  Det Datavetenskapliga Klubbverket är ansvariga för att:
  \begin{itemize}
    \item att se till att relevanta interna styrdokument upprättas och följs
    \item att anordna fester och andra sociala evenemang
    \item att genom kontakt med andra sektionsföreningar anordna större evenemang
    \item att anordna anmälan till aktiviteter utomstäders
  \end{itemize}
  
  \subsection{Övriga utskott}
  Det åligger styrelsen att vid behov utse andra eventuella representanter till utskott där datavetare skall representeras. Dessa kan väljas på stormöte eller styrelsemöte.}

\section{Övriga förtroendeuppdrag}
\subsection{Revisorn}
  \begin{itemize}
    \item att kontinuerligt under verksamhetsåret föra revision över föreningen och dess utskott
    \item att informera föreningens styrelse om resultatet av denna revision
    \item att innan och under verksamhetsåret informera de förtroendevalda hur revisionen kommer att genomföras och vilka krav som ställs på bokföring och verksamhet
  \end{itemize}
  
  \subsection{Valberedningen}
  Det åligger valberedningen att bereda val, uppmuntra till kandidatur och finna
  lämpliga kandidater till utlysta poster. Sammankallande till valberedningen
  väljs under ett stormöte till och med nästa ordinarie stormöte.
  \subsection{Sportansvarig}
  Sportansvarig är ansvarig för:

  \begin{itemize}
    \item att tillsammans med andra sektioner inom UTN anordna Ångströmsmästerskapen
    \item att främja sport inom sektionen
    \item att information om sportsrelaterade evenemang når ut till sektionens  medlemmar
    \item att köpa in, förvalta samt vårda sektionens sportutrustning
  \end{itemize}
\end{document}

%%% Local Variables:
%%% mode: latex
%%% TeX-master: t
%%% End:
